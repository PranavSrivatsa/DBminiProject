\chapter{Description of Tools and Technologies}

\section{PostgreSQL}
PostgreSQL is a powerful, open source object-relational database system that uses and extends the SQL language combined with many features that safely store and scale the most complicated data workloads.
PostgreSQL has earned a strong reputation for its proven architecture, reliability, data integrity, robust feature set, extensibility, and the dedication of the open source community behind the software to consistently deliver performant and innovative solutions. PostgreSQL runs on all major operating systems, has been ACID-compliant since 2001, and has powerful add-ons such as the popular PostGIS geospatial database extender. It is no surprise that PostgreSQL has become the open source relational database of choice for many people and organisations.

\thispagestyle{fancy}

\section{Python Flask}
Flask is a microframework for Python, written in Python based on Werkzeug, Jinja 2 and good intentions.
It is classified as a microframework because it does not require particular tools or libraries.It has no database abstraction layer, form validation, or any other components where pre-existing third-party libraries provide common functions. However, Flask supports extensions that can add application features as if they were implemented in Flask itself. Extensions exist for object-relational mappers, form validation, upload handling, various open authentication technologies and several common framework related tools. Extensions are updated far more regularly than the core Flask program.Flask is commonly used with MongoDB, which gives it more control over databases and history.
Applications that use the Flask framework include Pinterest,LinkedIn,and the community web page for Flask itself.

\thispagestyle{fancy}

\section{Flask-SQLAlchemy}
SQLAlchemy provides "a full suite of well known enterprise-level persistence patterns, designed for efficient and high-performing database access, adapted into a simple and Pythonic domain language". SQLAlchemy's philosophy is that relational databases behave less like object collections as the scale gets larger and performance starts being a concern, while object collections behave less like tables and rows as more abstraction is designed into them. For this reason it has adopted the data mapper pattern (similar to Hibernate for Java) rather than the active record pattern used by a number of other object-relational mappers.However, optional plugins allow users to develop using declarative syntax.

\thispagestyle{fancy}
